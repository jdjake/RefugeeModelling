\documentclass{article}

\pagestyle{myheadings}

\usepackage{graphicx}
\usepackage{amsmath}
\usepackage{cite}
\usepackage{tikz}
\usepackage{verbatim}
\usepackage[active,tightpage]{preview}
\usepackage{fancyheadings}
\usepackage{graphicx}
\usepackage{lastpage}

\pagestyle{fancy}

\fancyhead[C]{Team 53506}
\fancyhead[R]{Page \thepage\ of \pageref{LastPage}}
\fancyfoot{}

\usetikzlibrary{arrows}

\title{Safer Migration Strategies For Refugee Populations}
\author{Team 53506}

\begin{document}

\pagenumbering{gobble}
\maketitle
\tableofcontents

\maketitle

\newpage

\pagenumbering{arabic}

\section{Introduction}

Humanitarian organizations that aim to settle refugees in Europe are currently facing new challenges because refugees are coming from outside Europe in larger numbers than ever before\footnote{http://www.bbc.com/news/world-35091772}. One of these organizations, the office of the United Nations High Commissioner for Refugees (UNHCR) was initially formed in 1950 with the intent to help Europeans displaced by World War II\footnote{http://www.unhcr.org/pages/49c3646cbc.html}. Since its inception the UNHCR has provided aid to refugees, internally dispaced people, stateless people, and asylum seekers from emergencies originating within Europe and, increasingly, outside of Europe.\footnote{http://www.unhcr.org/pages/49c3646cbc.html}.

Conflict and poor governance are seen as the main reasons that people become refugees\footnote{http://www.bbc.com/news/world-35091772}. The 1951 refugee convention defines a refugee as ``owing to a well-founded fear of being persecuted for reasons of race, religion, nationality, membership of a particular social group or political opinion, is outside the country of his nationality, and is unable to, or owing to such fear, is unwilling to avail himself of the protection of that country''\footnote{http://www.unhcr.org/pages/49da0e466.html}, rather than a migrant, who is someone moving from one country to another without refugee status. According to data from the UNCHCR Global Trends 2014, ongoing conflicts in Syria and Afghanistan are the largest source of refugees\footnote{$http://ec.europa.eu/echo/files/aid/countries/factsheets/thematic/refugees_en.pdf$}. When considering how to solve the refugee crisis, it is important to pair dealing with the root causes with safe and efficient relocation of refugees.

Immigrants travel multiple routes -- beginning at the middle east, and travelling through the West, East, and Central Mediterranean, the West Balkans, the Eastern Borders, and from Albania to Greece.
CITE TRAVEL MAP

\begin{center}
\includegraphics[scale=0.5]{travelmap}
\end{center}

\begin{enumerate}
    \item The majority of refugees resettle in the middle eastern countries of Jordan, Lebanon, and Turkey.
    \item 1,015,078 refugees arrived by sea in 2015.
    \item 942,000 refugees sought asylum in the European Union in 2015\footnote{Eurostat}.
    \item 315,000 have sought asylum in Germany, yet more than 1 million have been counted in Germany's EASY system\footnote{http://www.bbc.com/news/world-europe-34131911}.
    \item 174,055 applied in Hungary.
\end{enumerate}

The political climate surrounding refugee aid in Europe is highly variable and subject to rapid change. Poorly managed camps in Hungary have damaged the reputation of migrants, leading to a surge in the popularity of the radical nationalist Jobbik party\footnote{http://www.bbc.com/news/world-europe-34280460}. The way countries will handle immigration crises cannot be deterministically modeled, so this is considered to be an exogenous factor. In other countries, such as Germany, strong moral foundations protect the influx of immigration\footnote{http://www.bbc.com/news/world-europe-33700624}, with little (and strongly opposed) backlash at the new immigration policies. In addition, unpredicted disasters such as the Paris attacks may threaten refugee relations with their host countries. In November 2015, the incoming Polish Minister for European Affairs said ``we will accept refugees only if we have security guarantees'' \footnote{http://www.bbc.com/news/world-europe-34826438}. The Paris attacks deepened the level of mistrust towards refugees across Europe, since it was believed that the terrorists snuck into the country with refugees. Regardless of the truth of these facts (the only known attackers are French and Belgian residents), the coinciding events were treated as such, and has lead to border problems in the country. Although the only known terrorists were French and Belgian residents, if there is increased mistrust towards refugees that may lead to changes in publically acceptable refugee policy.

We have been tasked with building a model that will help develop a better understanding of the factors involved with facilitating the movement of refugees from their countries of origin into safe haven countries. In doing this, we are attempting to determine the safest and most efficient routes that refugees should take and how many refugees should travel along those routes at a time. The numbers of refugees that should take any given route will relate to the capacity of possible destinations within Europe and the number of refugees within the system. 

Any proposal regarding large volumes of refugee movement should be considered a short term proposal. The resources available for refugee aid are finite, and the impetus for humanitarian aid may decrease over time, causing assumptions that are used to determine optimal travel routes to become inaccurate over longer time periods. The only way to permanantly ease the migrant situation in Europe is to end the conflicts that make people flee their countries in the first place. Therefore, we should only project our models into short time periods in the future.


This problem breaks down into multiple problems.

\begin{enumerate}
    \item {\bf What are the factors involved with moving refugees?}

    The UN asks us to prioritize the health and safety of refugees. What attributes enable or inhibit the safe and efficient movement of refugees? We consider the total numbers of refugees in the system and at each node, entry points for refugees, possible routes, popularity and capacity of those routes, length of routes, mode of transportation along the route, infrastructure for accommodation along routes, and capacity of European countries to recieve refugees to be 

    The total number of refugees entering the system and at each node, the entry points for refugees and capacity of European countries to recieve countries are important variables within the system. The entry points for refugees, possible routes, the popularity and capacity of those routes, the length of routes, mode of transportation and availabilty of infrastructure for accomodation along those routes are parameters to be considere in the system.

    Safety will be optimized by minimizing the risk that any individual may not complete a route. We define a measure of risk that incorporates information about liklihood of illness, death (i.e. drowning on water routes), and other dangers of travel that may be exacerbated for routes with high throughput for a given route. We assume that low densities have low risk and that high densities have high risk. In the absence of data describing this relationship further, for simplicity we assume that population density on a route has a positive linear relationship to risk. If more data becomes available, then a more realistic density-risk function may be substituted into the model.

    To assess the efficiency of a route, one may consider a case with high numbers of migrants to determine where bottlenecks occur and the factors that are slowing down flow at that point on the graph. Efficiency can be incorporated into the selection of an optimal route by favouring shorter routes or including a penalty for slower travel and for long waits in camps before reaching the end destination and exiting the system. Policy that prevents a high number of individuals existing for long periods of time without settling in a country should be preferred.

    Assumptions: All migrants who make it to their destination travel at the average rate. Due to our choice of a linear programming model, all relationships are linear. 

    \item {\bf How do we gain a better understanding of the factors?}

    Create a model of optimal refugee movement, considering accessibility of transport, safety of route, and resource capacities of countries. Use metrics considered to predict the number of refugees that are to be moved, as well as the rate and point of entry necessary to accomodate their movement. Explain new elements you have incorporated into the migration process.

    It should be noted that we must account for dynamically changing environmental factors. Our model should exhibit a use of endogenous change (prepositioning and allocating resources) to form the best route planning approach. Exogenous parameters (unavoidable, unpredictable events, like the terrorist attack in France) must also be considered.

    Graph with connected nodes. Known number of people at a number of entry points. Need to optimize distribution.

    Modelling methods considered:
    Stochastic Differential Equations are used in the stock market because individual factors cannot be influenced. This may be a way to realistically model the randomness of individual refugee movement. However, this type of model is difficult to optimize, which may make it difficult to meet our objective to optimize refugee movement. The use of stochastic programming may allow for optimization of the model. Stochastic effects are generally more important when you are dealing with small population sizes\footnote{Vries {\em et al.} 2006}. Since stochastic effects will likely have little impact on the overall movement of large groups of refugees along major migration routes, it is reasonable to model this situation using deterministic models. 

    Propogater models

    Modelling refugee movement may be treated as a social network problem, with multiple sources and sinks. Travel through the system may be described using fluid dynamics with partial differential equations describing each time step. One way to model exogenous events would be to 

    Create a graph where edges describe travel routes and vertices represent various locations that refugees may travel through and to.

    What do we know about our system?
    How much feedback exists in our system? Is it a closed system? 
    Let us consider a model that describes refugees that are not settled. This includes migrants who are residing in refugee camps in Northern Africa, migrants who are in transit, and migrants who have reached a destination, but do not have accepted migrant status (?) and have not been officially settled in the country. If a migrant becomes officialty settled then they have left the system. 

    Starting with a basic model, each possible country near Europe where there may be a substantial refugee population is represented as a vertex on a graph. The edges between vertices represent the connectivity between locations for refugees. Rates of travel along edges may vary based on qualities of the travel route including capacity, distance, modes of transportation available and risk to migrants. Once these variables and parameters are related on a graph, we need to learn about the dynamics of the system.


    One way to inform the movement of refugees from their country of origin into safe haven countries is to learn about how our system behaves when safety and efficiency are optimized. Safety and efficiency are optimized when risk is minimized and \underline{\ \ \ \ }, respectively. Risk and \underline{\ \ \ \ } can be combined linearly, resulting in an overall measure that determines We chose to use linear programming (as in \cite{bertsekas}) to optimize our system because it can always be solved. 

    Question?

    \item {\bf Propose a set of policies to the UN}

    Write a report to the UN, proposing a set of policies to enact which will support the conditions for optimal migration (optimal to the UN's views).
\end{enumerate}

The 1951 refugee convention defines a refugee to be ``owing to a well-founded fear of being persecuted for reasons of race, religion, nationality, membership of a particular social group or political opinion, is outside the country of his nationality, and is unable to, or owing to such fear, is unwilling to avail himself of the protection of that country''\footnote{http://www.unhcr.org/pages/49da0e466.html}, rather than a migrant, who is just someone moving from one country to another.

Immigrants travel multiple routes -- beginning at the middle east, and travelling through the West, East, and Central Mediterranean, the West Balkans, the Eastern Borders, and from Albania to Greece.

\includegraphics{travelmap}

\begin{enumerate}
    \item The majority of refugees resettle in the middle eastern countries of Jordan, Lebanon, and Turkey.
    \item 1,015,078 refugees arrived by sea in 2015.
    \item 942,000 refugees sought asylum in the European Union in 2015\footnote{Eurostat}.
    \item 315,000 have sought asylum in Germany, yet more than 1 million have been counted in Germany's EASY system\footnote{http://www.bbc.com/news/world-europe-34131911}.
    \item 174,055 applied in Hungary.
\end{enumerate}

Illegal immigration is dangerous, both for political relationships ensuring sustained immigration, and the danger to the immigrants of crossing the border. Poorly managed camps in Hungary damaged migrants reputation, leading to a surge in the popularity of the radical nationalist Jobbik party\footnote{http://www.bbc.com/news/world-europe-34280460}. It is difficult to control the way countries will handle the immigration crises, so this is an exogenous factor. In other countries, such as Germany, strong moral foundations protect the influx of immigration\footnote{http://www.bbc.com/news/world-europe-33700624}, with little (and strongly opposed) backlash at the new immigration policies. The Paris attacks threatened relations about refugees. A Polish minister stated that ``we were too idealistic'' \footnote{http://www.bbc.com/news/world-europe-34826438}. The attack deepened the level of insecurity across Europe, since it was believed that the terrorists snuck into the country with refugees. Regardless of the truth of these facts (the only known attackers are French and Belgian residents), the coinciding events were treated as such, and has lead to border problems in the country.

The difference in each country's utility to hold refugees is integral to how important it is to uphold their immigration policy. Economically, immigration should be beneficial to host countries in the long term. A well defined immigration policy, distributed across the continent, should not be a problem

\begin{enumerate}
    \item Our best evidence suggests that immigration is usually economically beneficial for host countries. The majority of refugees arriving on European shores are able-bodied and unlikely to be an exception to this general rule. So the best way for Europe to help would be to offer immediate legal residency and access to labour markets. It might be politically expedient to restrict access to some welfare benefits but most migrants will be keen to work regardless\footnote{http://www.telegraph.co.uk/news/worldnews/europe/11845205/Why-do-refugees-and-migrants-come-to-Europe-and-what-must-be-done-to-ease-the-crisis.html}.
\end{enumerate}

Note that our proposal is a short term proposal. The only way to permanantly ease the migrant situation in Europe is to end the conflicts that make people flee their countries in the first place. Therefore, we should only project our models into short time periods in the future (5 years?). How much do refugees move once they enter countries? Evidence to suggest immigrants attempt to flee Hungary and enter Germany.

Things to research for tomorrow:
\begin{enumerate}
    \item History, pertaining to dangers and effects of certain decisions on immigration policy.
    \item The UN's framework and objectives for the goals of refugee immigration.

    Under the general UN assembly, a refugee has the unconditional right to return home.

    \item How NGOs fit into the immigration picture.

    s

    \item How much does immigration cost?

    Typically, the total cost for processing and accommodating asylum seekers can be in the range of €8 000 and €12 0000 per application for the first year, although the figure may be much lower for fast track processing\footnote{http://www.oecd.org/migration/How-will-the-refugee-surge-affect-the-European-economy.pdf -- GOOD ECONOMIC PAPER}. 4 years after social assistance, 75\% of immigrants moved out of social assistance.

    \item Look up the application for official refugee status.

    Look in paper\footnote{$http://fra.europa.eu/sites/default/files/fra-focus_02-2015_legal-entry-to-the-eu.pdf$} and also\footnote{$https://www.hrw.org/report/2015/11/16/europes-refugee-crisis/agenda-action$}.

    The first step for a refugee is to arrive and register in a UNHCR refugee camp outside of Syria. The UNHCR then refers those who pass the first stage of vetting to the U.S. government refugee process (as described above). The National Counterterrorism Center, the Terrorist Screening Center, the Department of Defense, the FBI, Department of Homeland Security, and the State Department use biometrics and biographical information gleaned through several interviews of the refugee and third-party persons who know him or could know him to make sure applicants really are who they claim to be, to evaluate their security risk, and to investigate whether they are suspected of criminal activity or terrorism.  Numerous medical checks are also performed.  During this entire screening process, which takes about three years for Syrians, the refugee has to wait in the camp. If there is any evidence that the refugee is a security threat, he or she is not allowed to come to the United States\footnote{http://www.cato.org/blog/syrian-refugees-dont-pose-serious-security-threat}.

    \item How many immigrants do we need to accomodate?

    3 Million refugees to enter the EU by the end of 2016\footnote{http://www.independent.co.uk/news/world/europe/eu-expecting-another-3-million-refugees-migrants-before-end-of-2016-a6722096.html}.

    \item What Routes do immigrants take - what type of transportation are they taking?

    \item Where are immigrants trying to go?

    \item Quotas for countries.
\end{enumerate}

$http://www.pewresearch.org/fact-tank/2015/04/24/refugees-stream-into-europe-where-they-are-not-welcomed-with-open-arms/ft_15-04-22_eu-immigration/$


\section{Refugee Movement Optimization}

Our goal is to identify optimal travel routes for refugees seeking asylum. Ongoing conflicts in Syria, Afghanistan, the Middle East, and Africa are critical contributors\cite{refugeefactsheet}. Data shows most of these refugees flee to neighboring Middle Eastern countries, and on to Europe\cite{refugeefactsheet}. Our investigation focuses on these groups, and their transportation to the most popular states. To guide transportation, we optimize a refugee transportation model, from the conflicts discussed to the Middle East and Europe. Economically, it is unfeasible to transport large numbers of refugees by plane or boat, so we will not address refugee movement in this model.

Obvious by our choice of words, we consider an optimization problem. Under realistic constraints, which prevent perfect migration, a best achievable situation must be achieved. Most optimization problem are computationally unfeasible. Therefore, we restrict our model to linear and quadratic programming methods. Each route will be associated with an allocation of refugees. The UNs primary goal in this matter is to safeguard the well-being of refugees during the journey \cite{UNStatement}. Such safety concerns will be incorporated into our model as a risk parameter, which we attempt to minimize. Factors such as overcrowding, route danger, and route length will contribute most to this risk. Sometimes, it may be worth a risky journey to end up in a country which is least risky to live in.

Statistics provided by REPUTABLE SOURCE enable us to approximate the refugee capacity of each country we are considering. By additional analysis of ANOTHER REPUTABLE SOURCE, we can estimate the projected refugees produced in 2016. Details of such statistics used can be found in table INSERT TABLE HERE PLEASE. Our model assumes the use of such statistics, provided as a parameter to the system. We shall denote the capacity of a country $w$ as $C_w$, and the refugee productions of a country $v$ as $R_v$. The model detailed assumes that $\sum R_v \leq \sum C_w$, to obtain a feasible solution to the optimization procedure. Of course, as we can see from the European reaction to the refugee crisis, there may be a much greater demand of resources than can be accomodated. Regardless of a country's refusal to admit more refugees, desparate refugees will find a way to flee their home country, though perhaps to a subacceptable destination. To model this, we place `substandard' camps with capacities large enough to contain what other countries are unable to maintain.

We begin by forming a directed graph, whose nodes are countries, and whose edges are possible paths for refugees to take to obtain asylum. We consider any particular refugee's journey to be a simple path in the graph, because refugees are unlikely to take cyclic routes. More importantly for us, the UN should not guide refugees to take such non-optimal routes. For each simple path $(v_1, \dots, v_n)$, we allocate a number $x_{(v_1, \dots, v_n)} \in \mathbf{R}$, which represents the approximate number of refugees travelling along that path. The constraints are summarized by the equations below.

\begin{enumerate}
    \item (No Refugee Left Behind) For every country $v$ producing refugees,
    %
    \[ \sum_{\substack{(v_1, \dots, v_n) \\ v_1 = v}} x_{(v_1, \dots, v_n)} = R_v \]
    %
    Where $R_v$ is the number of refugees exiting the country $v$.

    \item (Bounded Capacity) For each asylum country $w$,
    %
    \[ \sum_{\substack{(v_1, \dots, v_n) \\ v_n = w}} x_{(v_1, \dots, v_n)} \leq C_w \]
    %
    Where $C_w$ is the capacity of the country $w$.
\end{enumerate}

We formulate risk as a quadratic functional, with certain quantified factors described below.

\begin{enumerate}
    \item {\it The risk of a route}. Each edge $(v,w)$ in the graph is associated with a certain `risk constant' $K_{(v,w)}$ that represents the probability of death for a single refugee travelling along that edge. If $n$ refugees travel along this path, then the expected number of deaths will be $n K_{(v,w)}$. The risk of death travelling along a certain path $(v_1, \dots, v_n)$ is then compounded. By basic laws of probability, we have
    %
    \[ \mathbf{P}(\text{immigrant dies on}\ (v_1, \dots, v_n)) = \sum_{i = 1}^{n-1} \mathbf{P}(\text{immigrant dies on}\ (v_i, v_{i+1})) \]
    %
    We obtains an analogous equation for the risk constant, where the chance an immigrant dies on an edge is the product of the probability that the immigrant does not die on all previous edges, compounded with the probability that the immigrant dies on the final edge.
    %
    \[ K_{(v_1, \dots, v_n)} = \sum_{i = 1}^n \left( \prod_{j = 1}^{i-1} \left(1 - K_{(v_j,v_{j+1})} \right) \right) K_{(v_i, v_{i+1})} \]

    \item {\it Route overcrowding}. In an overcrowded group of refugees, disease, violence, and crime are likely to cause harm, so we must manage traveller density in order to reduce risk. We assume the number of interactions within a population is quadratically proportional to population density. It follows that the risk factors of disease, violence and crime, all due to population interaction, are also quadratically propertional to density. We assume that there exists a constant $B_{(v_1, \dots, v_n)}$ for each route $(v_1, \dots, v_n)$ such that the health compromising rates for a group of $n$ people are proportions to $B_{(v_1, \dots, v_n)} n^2$. In general, we shall have to consider `correlation rates', since two paths $(v_1, \dots, v_n)$ may coincide in such a way to cause congestion, leading to a rate of the form $B_{(v_1, \dots, v_n), (w_1, \dots, w_m)} n m$, where $m$ is the group travelling along a path $(w_1, \dots, w_m)$. Of course, for routes that never coincide, this value may be zero. For notational homogeneity, we shall also denote $B_{(v_1, \dots, v_n)}$ by $B_{(v_1, \dots, v_n), (v_1, \dots, v_n)}$. Derivation of these constants will be obtained in the next section.
\end{enumerate}

Our risk functional and constraints form a quadratic program. To summarize, our quadratic program is described in the following form.
%
\begin{align*}
    &\text{min} \sum_{(v_1, \dots, v_n)} K_{(v_1, \dots, v_n)} x_{(v_1, \dots, v_n)}\\
    & \ + \sum_{\substack{(v_1, \dots, v_n)\\(w_1, \dots, w_m)}} B_{(v_1, \dots, v_n) (w_1, \dots, w_m)} x_{(v_1, \dots, v_n)} x_{(w_1, \dots, w_m)}\\
    &\text{s.t. for each source $v$ and sink $w$,}\\
    &\ \ \ \ \ \sum_{\substack{(v_1, \dots, v_n) \\ v_1 = v}} x_{(v_1, \dots, v_n)} = R_v\\
    &\ \ \ \ \ \sum_{\substack{(v_1, \dots, v_n) \\ v_n = w}} x_{(v_1, \dots, v_n)} \leq C_w
\end{align*}
%
Any of your favourite quadratic optimization methods will find an optimum solution to this problem.

\section{Deriving Path Correlation Coefficients}

In its current form, our linear program does not take into account the intersection of paths formed by travelling refugees. It also assumes that groups of refugees travel deterministically. This obviously does not hold in any practical sense -- No two refugees are completely alike, and react differently in response to different events.

We could discard this assumption as a quirk of the model, besides the fact that it becomes a problem when trying to find the path correlation coefficients $B_{(v_1, \dots, v_n) (w_1, \dots, w_m)}$. Consider two curves, together with parameterizations $c$ and $c'$ with unit velocity. These curves represent one of the possible paths that refugees can take. When we model the movement of refugees as a single point on the line, then the movement of two groups of refugees coincides when $c(t) = c'(t)$. If the traces of $c$ and $c'$ intersect, but hit all points in the intersection at different times, then our model would determine that these groups never meet each other. On the other hand, we cannot just take the traces as evidence for congestion, for we would like to increase the importance of an intersection $c(t) = c(t')$ when $t$ and $t'$ are very close, and decrease the importance when the time points are far apart. After a long time, most refugees will have moved away from this point, so congestion is near negligable. We solve this problem by taking a stochastic movement along these curves, determining the expected intersection probability when two groups encounter one another.

\begin{figure}[h]
\begin{center}
\includegraphics[scale=0.6]{differentpaths}
\caption{Allocating a group of immigrants to either of these paths should congest the route of the other path. Image obtained from \cite{twocurve}.}
\end{center}
\end{figure}

To accomodate the random motion of population migration, we apply the theory of stochastic processes, as developed in \cite{lawler}. Begin by making the following assumptions
%
\begin{enumerate}
    \item {\it Every immigrant starts the route at the start point.} Technically, immigrants could leave from various parts of a certain country, or join a group of refugees after they have already begun travelling.
    \item {\it The movement of immigrants, as a function of time, is continuous}. Physically, this assumption should always be satisfied.
    \item {\it The process of immigration movement is Markovian.} Past movement of a certain immigrant cannot predict future movement of that immigrant, aside from where the immigrant is at a certain timepoint. This assumption is to obtain a feasible solution. Though for a specific immigrant this assumption will not be satisfied, when we take statistical averages of immigration this assumption should at least be a good approximation.
    \item {\it Immigrant travel spreads according to a normal distribution}. This represents a `diffusion' of immigrants over time as they travel to their destination, which converges to the final destination asymptotically. By the law of large numbers, the distribution of immigrants should be sufficiently approximated by a normal distribution, so this assumption is not too sensitive to change.
\end{enumerate}
%
Now we calculate the coefficient of immigration. First, we pull back the curve $c$ to its domain $[0,A]$, and extending the definition of $c$ to $\mathbf{R}$, by definining $c(A + t) = c(A)$, $c(-t) = 0$ for $t \geq 0$. We shall place a stochastic process on the interval. The assumptions above uniquely determine the structure of the process. Since $c$ has unit velocity at all time points, we can describe the process $X_t$ which models the movement of immigrants via the stochastic equation
%
\[ X_t = \varepsilon W_t + t \]
%
where $\varepsilon$ is a small constant, and $W_t$ is standard brownian motion. We construct an independant task for each path $(v_1, \dots, v_n)$, represented by a curve $c$. If $X_t$ is such an equation, then $c(X_t)$ gives us a stochastic process on the path in the graph. We wish to measure, in some capacity, the `population correlation' between the stochastic processes $c(X_t)$ and $c'(Y_t)$. We cannot use the probability that two particular instances of the motion meet at a certain time point, since $\mathbf{P}(c(X_t) = c'(Y_t)) = 0$ for all values of $t$. The best we can do is approximate two populations coinciding; fixing a small $\varepsilon'$, we consider the `intersection measure'
%
\[ B_{(v_1, \dots, v_n), (w_1, \dots, w_m)} = \int_0^\infty \int_0^1 \mathbf{P}\left(c'(Y_t) - \varepsilon' < c(X_t) < c'(Y_t) + \varepsilon'\right)\ dY_t\ dt \]
%
which sums up the possible instances of intersection. Theoretically, this is an approximation. Practically, this method should perform well in practice; really, two people can never be in the {\it exact} same location at a particular time, so our model really does model the correct situation, provided we pick $\varepsilon'$ so that `vicinity' is small enough for two populations to affect one another, and big enough to obtain accurate interaction estimates. To calculate the integral, we perform some method of approximation. Due to the arbitrary nature of the parameterizations, an analytic form is impossible to obtain.

Our method above also allows us to extend our model to additional timeframes. We can now consider waves of immigrants, which begin at different time periods. Using the correlation coefficient determination method above (modified to take into account different time parameterizations).

\section{Metrics of Refugee Crisis}

\bibliographystyle{plain}
\bibliography{refugee_citations}

\end{document}